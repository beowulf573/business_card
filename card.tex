%
% Originally in http://www.debian.org/~jgg/card.tex
%  Debian GNU/Linux developer business card template
%
%  This template is vaguely based on several similar ones I have
%  found elsewhere on the web; it uses a LaTeX ``picture'' environment
%  to position the text all around an individual card, and wraps that
%  in a bigger ``picture'' which renders 10 cards on the page in a
%  layout suitable for printing on Avery pre-punched business card
%  stock.
%
%  The modified cut marks are aligned outside the actual card. When cutting
%  this makes sure that there is no slight black line on the top or the
%  bottom of the cards. The best way to cut them is using a cutting wheel
%  [looks like a pizza cutter but is extrmely sharp] and a plastic pad. Use
%  a rular to guide the cutter. Cut the three vertical divisions first, but 
%  do not go past the top or bottom of the page. After that cut each card 
%  horizontally. If you have a sharp cutter you can cut more than one sheet
%  at once like this. The prefered type of paper is a heavy paper run through
%  a laser in the `straight through' path [usually in the manual feed and
%  out the face up tray on the back]

\documentclass[11pt]{article}
\usepackage[dvips]{graphicx}
%
%  I personally like the look of the CM fonts, especially compared to
%  plain old times/helvetica.  Of course, you can select a different
%  set of fonts here if you like -- Adobe Myriad looks good on a card,
%  and for mixed serif/sans the Stone family hangs together nicely,
%  but unfortunately those aren't part of TeX.
%
%\usepackage{times}
%\usepackage{ccfonts}
\usepackage[osf,sf]{engserif} % 5es
%
\pagestyle{empty}
%
%  Before printing a final copy, you will probably have to tweak these
%  values to get the cards to properly line up with the cut marks.
%  Avery provides a sizing overlay with their cards; if you're having
%  these professionally printed, get the appropriate dimensions from
%  your printer [and you might need to adjust the multiput commands
%  below!].
%
\setlength{\textwidth}{7in}
\setlength{\hoffset}{0in}
\setlength{\voffset}{0in}
\setlength{\oddsidemargin}{-.8in}
\setlength{\evensidemargin}{-.8in}
\setlength{\textheight}{10in}
\setlength{\topmargin}{-0.87in}

%
%  Text variables for the card.  If you want to change how these are
%  rendered, see below.
%
%\newcommand{\xname}{\small Mag. (FH) \large Martin W\"urtele}
%\newcommand{\xdebian}{Debian GNU/Linux}
%\newcommand{\xdeburl}{http://www.heorot.org}
\newcommand{\xname}{Eddie McCreary}
\newcommand{\xtitle}{Developer}
\newcommand{\xaddress}{16101 Congo Lane}
\newcommand{\xaddressb}{Houston, TX 77040}
\newcommand{\xphone}{(281) 851-2417}
\newcommand{\xemail}{}
\newcommand{\yemail}{lazarus@heorot.org}
\newcommand{\xurl}{http://www.heorot.org}
%
%  The PGP keys are printed in a tiny little font along the bottom
%  edge of the card.  If you come up with a better way to fit them in,
%  let me know.
%
% You can get the fingerprint by doing : 'pgp -kvc <userid>'
\newcommand{\xpgpkeyidA}{95262F3E}
\newcommand{\xpgpbitsA}{1024D}
\newcommand{\xpgpfingerprintA}{4C06 572C 882E 9682 2C23  5FD4 7BEF ECE9 9526 2F3E}
\newcommand{\xpgpkeyA}{\xpgpbitsA/\xpgpkeyidA: \xpgpfingerprintA}

% There is room for 3 keys, uncomment the bit below and duplicate the above

\newcommand{\pgpfont}{\tt \fontsize{.08in}{.096in}\selectfont}

\newcommand{\xpgpkeyidB}{3E8DCCC0}
\newcommand{\xpgpbitsB}{1024D}
\newcommand{\xpgpfingerprintB}{30DC 1D28 1D79 32F5 5E67  3ABB 28EE B35A 3E8D CCC0}
\newcommand{\xpgpkeyB}{\xpgpbitsB/\xpgpkeyidB: \xpgpfingerprintB}

%
\begin{document}
\setlength{\unitlength}{1mm}
\begin{picture}(178,253)(-7,-1)

  % Vert ticks along the bottom
  \put(0,-1){\line(0,-1){3}}
  \put(89,-1){\line(0,-1){3}}
  \put(179,-1){\line(0,-1){3}}

  % Vert ticks along the top
  \put(0,255){\line(0,1){3}}
  \put(89,255){\line(0,1){3}}
  \put(179,255){\line(0,1){3}}

    \put(-1,-1){\line(-1,0){3}}
    \put(180,-1){\line(1,0){3}}

  \multiput(0,0)(0,51.0){5}{%
    % Ticks on the side of each card
    \put(-1,51){\line(-1,0){3}}
    \put(180,51){\line(1,0){3}}
    \multiput(0,0)(89,0){2}{%
      \begin{picture}(89,51)(0,0)
	%
	%  Line across the top of the card.
	%
        \put(3,48){\line(1,0){83}}
	%
%        \put(13.5,42){\large\textsc{\xdebian}}
%        \put(13.5,38){\scriptsize\xdeburl}


	\put(6,33){\textsf{\sc\large\xname}}
	\put(6,28){\textsf{\sc\small\xtitle}}


	% depending on lenth this needs a little manual tweeking, didn't find
	% a way to use right alignment within \put(){}
        \put(58.45,18){\textsf{\it\scriptsize \xemail}}
        \put(54.65,15){\textsf{\it\scriptsize \yemail}}
%        \put(45.4,12){\textsf{\scriptsize \xurl}}
        \put(50,12){\textsf{\it\scriptsize \xurl}}
	       
        \put(6,18){\scriptsize \xaddress}
        \put(6,15){\scriptsize \xaddressb}
	\put(6,12){\scriptsize \xphone}
        
	
	% For one key [1.2x]
        \put(6,6){\textsf{\pgpfont \xpgpkeyA}}
	
	% For two keys
        %\put(5,8){\textsf{\pgpfont \xpgpkeyA}}
        %\put(5,6){\textsf{\pgpfont \xpgpkeyB}}	
        %\put(5,4){\textsf{\pgpfont \xpgpkeyC}}	% For three keys
	
	%  Line across the bottom of the card.
	%
        \put(3,3){\line(1,0){83}}
      \end{picture}}}
\end{picture}
\end{document}

% last modified: Thu Nov 21 15:32:23 CET 2002
